\documentclass[a4paper,onesided,11pt]{report}
\usepackage{styles/teknofest_cip_tasarim}
\usepackage[utf8x]{inputenc} % To use Unicode (e.g. Turkish) characters
\renewcommand{\labelenumi}{(\roman{enumi})}
\usepackage{amsmath, amsthm, amssymb}
 % Some extra symbols
\usepackage[bottom]{footmisc}
\usepackage{cite}
\usepackage{layout}
\usepackage{hyperref}
\usepackage[notocbib]{apacite}
\usepackage{graphicx}
\usepackage{longtable}
\graphicspath{{figures/}} % Graphics will be here
\usepackage{mathtools}
%\usepackage[ngerman]{babel} % Or polyglossia
\usepackage[sfdefault]{carlito}
\usepackage{inconsolata}

%\usepackage{xtab}
\usepackage{multirow}
\usepackage{subfigure}
\usepackage{algorithm}
\usepackage{algorithmic}
\usepackage{draftwatermark}
\usepackage{tikz}

\usepackage{url}
\usepackage{xcolor}
%\hypersetup{
%    colorlinks,
%    linkcolor={red!50!black},
%    citecolor={blue!50!black},
%    urlcolor={blue!80!black}
%}
\usepackage[ddmmyyyy]{datetime}
\renewcommand{\dateseparator}{.}
%\pagestyle{empty}
%\includeonly{introduction} % To only process the given file
\makeatletter
\renewcommand*{\lay@value}[2]{%
  \strip@pt\dimexpr0.351459\dimexpr\csname#2\endcsname\relax\relax mm%
}
\makeatother
\newtheorem{thm}{Theorem}[chapter]
\newtheorem{prop}[thm]{Proposition}
\newtheorem{lem}[thm]{Lemma}
\newtheorem{cor}[thm]{Corollary}
% COVER PAGE
\teamname{...}
\projectname{...}
\applicationid{...}

\SetWatermarkText{\tikz{\node[opacity=0.2]{\includegraphics{filigran.jpg}};}}
%\SetWatermarkText{\includegraphics{filigran.jpg}}
\SetWatermarkAngle{0} 
\SetWatermarkLightness{1}

\begin{document}

\pagenumbering{roman}
\maketitlepage
\tableofcontents
\begin{symbols}
% The title will be typeset as "LIST OF SYMBOLS".
%
% Use a separate \sym command for each symbols definition.
% First, Latin symbols in alphabetical order
\sym{GF($p^{k}$)}{Galois Field with order $p^{k}$}
\sym{$V_{T}$}{Threshold voltage}
% 1 EMPTY LINE BETWEEN LATIN AND GREEK SYMBOLS GROUP!!!
\sym{}{}
% Then Greek symbols in alphabetical order
\sym{$\mu$}{Population mean}
\sym{ }{}

\end{symbols}

\begin{abbreviations}
 % Abbreviations in alphabetical order
\sym{ASIC}{Application Specific Integrated Circuit}
\sym{FPGA}{Field Programmable Gate Array}
\sym{SRAM}{Static Random Access Memory}
\sym{TSMC}{Taiwan Semiconductor Manufacturing Company}
\end{abbreviations}

\clearpage
\chapter{TEMEL TASARIM ÖZETİ}
\label{chapter:temel_tasarim_ozeti}
\pagenumbering{arabic}

Bu doküman 2022 TEKNOFEST Çip Tasarım Yarışması kapsamında hazırlanması beklenen Detaylı Tasarım Raporu (DTR) için kullanılması amacıyla hazırlanmıştır. İngilizce rapor için hem .tex dosyasının hem de .sty dosyasının güncellenmesi gerekmektedir. Örnek şekil ve çağırılışı şöyle olmaktadır: Şekil \ref{fig:yarisma_logosu}. 

\begin{figure}[htbp]
	\begin{center}
		\includegraphics[width=0.3\columnwidth]{cip_tasarim.png}
		\vskip\baselineskip % Leave a vertical skip below the figure
		\caption{Yarışma Logosu.}
		\label{fig:yarisma_logosu}
	\end{center}
\end{figure}

American Psychological Association (APA) standardına uygun referans vermek amacına hizmet eden örnek alıntılama biçimi \cite{Lamport1986}. Detaylar için apacite paketinin opsiyonları incelenebilir.


Örnek Tablo için Tablo \ref{table:ornek_tablo}'a bakılabilir. Caption'ların sonunda nokta olmalıdır.

\begin{table}[thbp]
%\vskip\baselineskip 
\caption[Örnek Tablo]{Örnek Tablo.}
\begin{center}
\begin{tabular}{|c|c|c|} \hline
 & \textbf{Sütun 1}& \textbf{Sütun 2}\\\hline
\textbf{Satır 1} & Veri 1 & Veri 2 \\\hline
\textbf{Satır 2} & Veri 3 & Veri 4 \\\hline
\textbf{Satır 3} & Veri 5 & Veri 6 \\\hline
\end{tabular}
\label{table:ornek_tablo}
\end{center}
\end{table}



Örnek dipnot kullanımı için\footnote{Örnek dipnot}  

Madde listesi ise aşağıdaki gibi hazırlanabilir.

\begin{itemize}
 \item Madde 1
\begin{enumerate}
 \item Alt madde
\end{enumerate}
\item Madde 2
\item Madde 3
\end{itemize}

Matematiksel işlemler \eqref{eq:Pearson_Correlation_Function}'taki gibi verilebilir.
\begin{equation} \label{eq:Pearson_Correlation_Function}
C (T, P) = \displaystyle \frac{\mu (TP)-\mu (T) \mu (P)}{\sqrt{\sigma ^2 (T) \sigma ^2 (P)}} \\
\end{equation}

\chapter{PROJE MEVCUT DURUM DEĞERLENDİRMESİ}
\label{chapter:proje_mevcut_durum_degerlendirmesi}

\chapter{PROJE DETAY TASARIMI}
\label{chapter:proje_detay_tasarimi}

\section{Sistem Mimarisi}
\label{section:sistem_mimarisi}

\section{Tasarım Detayı}
\label{section:tasarim_detayi}

\subsection{Çekirdek Tasarımı}
\label{subsection:cekirdek_tasarimi}

\subsection{Bellek Tasarımları}
\label{subsection:bellek_tasarimlari}

\subsection{Çevre Birimleri Tasarımları}
\label{subsection:cevre_birimleri_tasarimlari}

\chapter{ÇİP TASARIM AKIŞI}
\label{chapter:cip_tasarim_akisi}

\chapter{TEST}
\label{chapter:test}

\chapter{TAKIM ORGANİZASYONU}
\label{chapter:takim_organizasyonu}

\section{Takım Organizasyonu}
\label{section:takim_organizasyonu}

\section{Görev Dağılımı}
\label{section:gorev_dagilimi}

\chapter{İŞ PLANI ve RİSK PLANLAMASI}
\label{chapter:is_plani_ve_risk_planlamasi}

%\cite{*}
\bibliographystyle{apacite}
\renewcommand\bibname{KAYNAKÇA}
\bibliography{references}
\appendix
\chapter[ÖRNEK EK]{ÖRNEK EK}
Ekler referanslardan sonra gelmektedir.
%\clearpage
%\layout*
\end{document}